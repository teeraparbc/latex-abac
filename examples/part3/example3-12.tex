\documentclass[12pt, a4paper]{article}

\usepackage{makeidx}

\makeindex

\begin{document}

In the mathematical field of numerical analysis, interpolation\index{interpolation} is a method of constructing new data points within the range of a discrete set of known data points.

\newpage
One of the simplest methods is linear interpolation\index{interpolation!linear interpolation}.

Polynomial interpolation\index{interpolation!polynomial interpolation|(} is a generalization of linear interpolation\index{linear interpolation@{\bf linear interpolation}}. Note that the linear interpolant is a linear function.

\newpage

Polynomial interpolation\index{interpolation!polynomial interpolation|)} is a generalization of linear interpolation. Note that the linear interpolant is a linear function. We now replace this interpolant with a polynomial of higher degree.

In mathematics, extrapolation\index{extrapolation|see{interpolation}} is the process of estimating, beyond the original observation range, the value of a variable on the basis of its relationship with another variable.

\printindex

\end{document}